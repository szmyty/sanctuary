%! Author = Alan Szmyt
%! Date = 08/05/2024

% Preamble
\documentclass[../../main.tex]{subfiles}

% Document
\begin{document}

    % Section: The Push-to-Start and Push-to-Restart Paradox
    \section{The Push-to-Start and Push-to-Restart Paradox}

    In pursuing the ideal of a push-to-start system, I've often encountered the paradox of needing a push-to-restart mechanism. The ability to destroy and rebuild systems effortlessly is crucial in today's agile environments. This mirrors our personal and professional journeys—we must be willing to deconstruct and reevaluate our beliefs and methods to grow.

    This cyclical process of building, evaluating, and rebuilding is not a setback but an essential aspect of development. It aligns with the spiral concept, where each loop represents not a repetition but an elevation.

    \vspace{0.5cm}
    \noindent\includegraphics[width=\linewidth]{push_to_start.png}
    \newline
    \textit{Figure 4: The push-to-start system embodies the cyclical yet elevating nature of development.}
    \vspace{0.5cm}

\end{document}
