%! Author = Alan Szmyt
%! Date = 08/05/2024

% Preamble
\documentclass[../../main.tex]{subfiles}

% Document
\begin{document}

    % Section: The Evolution of the Developer's Role
    \section{The Evolution of the Developer's Role}

        The role of a developer has long been categorized by the term \textbf{full-stack developer}, typically someone proficient in both back-end and front-end technologies. However, with the ever-growing complexity of modern systems, this definition feels increasingly restrictive. Today's developers must navigate a broader and more integrated spectrum—ranging from infrastructure as code, cloud-native applications, CI/CD pipelines, and container orchestration, to user experience design.

        This evolving reality gives rise to a more fitting concept: the \textbf{full-loop developer}. Unlike the traditional full-stack model, the full-loop developer embodies a cyclical approach to development. They are engaged not just in building and maintaining code, but in continuously iterating through the entire \textbf{development lifecycle}—coding, testing, deploying, monitoring, and refining. The loop represents this continuous process of improvement and feedback.

        Yet, the idea of a simple loop is still incomplete. The loop implies a closed cycle, a repetition, but in reality, each iteration builds on the last, pushing us forward. Our skills and experiences grow with every pass, creating a sense of vertical progression, like a spiral that weaves upwards rather than merely cycling in place. In this way, the full-loop developer does not merely repeat the same steps but adds layers of expertise and understanding with every iteration, ascending to new levels of mastery.

        \figureimage[fig:loop]{loop_developer.png}{Figure 2: The full-loop developer transcends the traditional stack, cycling through all layers of technology while continuously building expertise and evolving.}
\end{document}
